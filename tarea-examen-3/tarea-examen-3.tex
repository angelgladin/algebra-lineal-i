%%%
 %
 % Copyright (C) 2020 Ángel Iván Gladín García
 %
 % This program is free software: you can redistribute it and/or modify
 % it under the terms of the GNU General Public License as published by
 % the Free Software Foundation, either version 3 of the License, or
 % (at your option) any later version.
 %
 % This program is distributed in the hope that it will be useful,
 % but WITHOUT ANY WARRANTY; without even the implied warranty of
 % MERCHANTABILITY or FITNESS FOR A PARTICULAR PURPOSE.  See the
 % GNU General Public License for more details.
 %
 % You should have received a copy of the GNU General Public License
 % along with this program.  If not, see <http://www.gnu.org/licenses/>.
%%%

%%%%%%%%%%%%%%%%%%%%%%%%%%%%%%%%%%%%%%%%%%%%%%%%%%%%%%%%%%%%%%%%%%%%%%%%%%%%%%%%%%%%%%%%%
\documentclass[letterpaper]{article}
\usepackage[margin=.75in]{geometry}
\usepackage[utf8]{inputenc}
\usepackage[spanish]{babel}
\decimalpoint

\usepackage{enumerate}
\usepackage{enumitem}
\usepackage{float}
\usepackage{dsfont}
\usepackage{amsmath,amsthm,amssymb}
%%%%%%%%%%%%%%%%%%%%%%%%%%%%%%%%%%%%%%%%%%%%%%%%%%%%%%%%%%%%%%%%%%%%%%%%%%%%%%%%%%%%%%%%%%%%%%%%5

%%%%%%%%%%%%%%%%%%%%%%%%%%%%%%%%%%%%%%%%%%%%%%%%%%%%%%%%%%%%%%%%%%%%%%%%%%%%%%%%%%%%%%%%%


%%%%%%%%%%%%%%%%%%%%%%%%%%%%%%%%%%%%%%%%%%%%%%%%%%%%%%%%%%%%%%%%%%%%%%%%%%%%%%%%%%%%%%%%%
\newcommand{\Z}{\mathbb{Z}}
\newcommand{\N}{\mathbb{N}}
\newcommand{\Q}{\mathbb{Q}}
\newcommand{\R}{\mathbb{R}}
\newcommand{\Po}{\mathds{P}}

\newcommand{\af}{\textbf{\underline{Afirmación}: }}
\newcommand{\cej}{\textbf{\underline{Contraejemplo}: }}

\DeclareMathOperator{\rank}{rank}
\DeclareMathOperator{\Span}{span}

\newtheorem*{sol}{Solución}

%%%%%%%%%%%%%%%%%%%%%%%%%%%%%%%%%%%%%%%%%%%%%%%%%%%%%%%%%%%%%%%%%%%%%%%%%%%%%%%%%%%%%%%%%

%%%%%%%%%%%%%%%%%%%%%%%%%%%%%%%%%%%%%%%%%%%%%%%%%%%%%%%%%%%%%%%%%%%%%%%%%%%%%%%%%%%%%%%%%
\begin{document}
\title{
        Universidad Nacional Autónoma de México\\
        Facultad de Ciencias\\
        Álgebra Lineal I\\
    \vspace{.5cm}
    \large
        \textbf{Tarea-Examen 3}
}
\author{
    Ángel Iván Gladín García\\
    No. cuenta: 313112470\\
    \texttt{angelgladin@ciencias.unam.mx}
}
\date{5 de junio de 2020}
\maketitle
%%%%%%%%%%%%%%%%%%%%%%%%%%%%%%%%%%%%%%%%%%%%%%%%%%%%%%%%%%%%%%%%%%%%%%%%%%%%%%%%%%%%%%%%%

\begin{enumerate}

%%%%%%%% 1
\item Considere matrices $A$, $B$ y $C$ de $n \times n$. Demuestre lo siguiente:
\begin{enumerate}[label=(\alph*)]
    \item El $\det(A) = \det\left( A^T \right)$
    \begin{proof}
    Si $A$ no es invertible, entonces $\rank(a) < n$. Pero $\rank(A^T) = \rank(A)$, y así $A^T$ no es invertible.
    Por tanto $\det(A^T) = \det(A)$ en este caso.

    Por otro lado, si $A$ es invertible, entonces $A$ es un producto de matrices elementales, digamos
    $A = E_m \cdots E_2 E_1$. Como $\det(E_i) = \det(E_i^T)$ para cada $i$, se tiene que.

    \begin{align*}
        \det(A^T)
            &= \det(E_1^T E_2^T \cdots E_M^T)\\
            &= \det(E_1^T) \cdot \det(E_2^T) \cdots \det(E_m^T)\\
            &= \det(E_1) \cdot \det(E_2) \cdots \det(E_m)\\
            &= \det(E_m) \cdot \det(E_2) \cdot \det(E_1)\\
            &= \det(E_m \cdots E_2 E_1)\\
            &= \det(A)
    \end{align*}

    \end{proof}

    \item Si $C$ se obtuvo de $A$ al cambiar el $i$-ésimo renglón (columna) por lo $j$-ésimo renglón
    (columna). Muestre que $\det(C) = -\det(A)$.
    \begin{proof}
    Sean los renglones de $A$ de la forma $a_1, a_2, \ldots, a_n$ y sea $C$ la matriz obtenida de $A$ de intercambiar
    los renglones $r$ y $s$, donde $r < s$. Así
    
    \[
        A = 
        \begin{pmatrix}
            a_1\\
            \vdots\\
            a_r\\
            \vdots\\
            a_s\\
            \vdots\\
            a_n
        \end{pmatrix}
        \quad \text{y} \quad
        C = 
        \begin{pmatrix}
            a_1\\
            \vdots\\
            a_s\\
            \vdots\\
            a_r\\
            \vdots\\
            a_n
        \end{pmatrix}
    \]
    
    Consideremos la matriz obtenida de $A$ de remplazar los renglones $r$ y $s$ por $a_r + a_s$.

    Se tiene que,

    \begin{align*}
        0
        &=
            \det\begin{pmatrix}
                a_1\\
                \vdots\\
                a_r + a_s\\
                \vdots\\
                a_r + a_s\\
                \vdots\\
                a_n
            \end{pmatrix} =
            \det\begin{pmatrix}
                a_1\\
                \vdots\\
                a_r\\
                \vdots\\
                a_r + a_s\\
                \vdots\\
                a_n
            \end{pmatrix} +
            \det\begin{pmatrix}
                a_1\\
                \vdots\\
                a_s\\
                \vdots\\
                a_r + a_s\\
                \vdots\\
                a_n
            \end{pmatrix}\\
        &=
            \det\begin{pmatrix}
                a_1\\
                \vdots\\
                a_r\\
                \vdots\\
                a_r\\
                \vdots\\
                a_n
            \end{pmatrix} =
            \det\begin{pmatrix}
                a_1\\
                \vdots\\
                a_r\\
                \vdots\\
                a_s\\
                \vdots\\
                a_n
            \end{pmatrix} +
            \det\begin{pmatrix}
                a_1\\
                \vdots\\
                a_s\\
                \vdots\\
                a_r\\
                \vdots\\
                a_n
            \end{pmatrix} +
            \det\begin{pmatrix}
                a_1\\
                \vdots\\
                a_s\\
                \vdots\\
                a_s\\
                \vdots\\
                a_n
            \end{pmatrix}\\
        &=
            0 + \det(A) + \det(C) + 0
    \end{align*}
    Por tanto $\det(C) = -\det(A)$.
    \end{proof}

    \item $\det(AB) = \det(A)\det(B)$.
    \begin{proof}
    Si $A$ no es invertible, entonces $AB$ no es invertible, entonces se sigue cumpliendo la igualdad
    $0 = \det(AB) = \det(A)\det(B) = 0$. Supongamos que $A$ es invertible, entonces existen operaciones
    elementales de renglones $E_k, \ldots, E_1$ tal que $A = E_k \cdots E_1$.

    Entonces,
    \begin{align*}
        \det(AB)
            &= \det(E_k \cdots E_1 B)\\
            &= \det(E_k) \det(E_{k-1} \cdots E_1  B)\\
            &= \det(E_k) \cdots \det(E_1)\det(B)\\
            &= \det(E_k \cdots E_1)\det(B)\\
            &= \det(A)\det(B)
    \end{align*}
    \end{proof}

    \item Sea $C$ una matriz obtenida a partir de $A$ al multiplicar por $c \in F$ un renglón. Muestre que
    $\det(C) = c \cdot \det(A)$.
    \begin{proof}
    TODO\footnote{No me dio tiempo porque tenía otras tareas.}
    \end{proof}
\end{enumerate}

%%%%%%%% 2
\item Demuestre que un sistema de ecuaciones lineales $Ax = b$ tiene solución si y sólo si $b \in R(L_A)$.
\begin{proof}
\hfill
\begin{itemize}
    \item[$(\Longrightarrow)$] Sea $A = (a_1, a_2, \ldots, a_n)$ una matriz donde $a_i$ es la $i$-ésima columna. Sea
    $s$ una solución,
    \[
        s =
        \begin{pmatrix}
            x_1\\
            x_2\\
            \vdots\\
            x_n
        \end{pmatrix}
    \]
    Sea $b = As = A_1s_1 + a_2s_2 + \ldots + a_ns_n$, así $b \in R(L_A)$.

    \item[$(\Longleftarrow)$] Supongamos $b \in R(L_A)$, entonces existe $b = As$, donde $s$ es una solución de
    $Ax = b$.
\end{itemize}
\end{proof}

%%%%%%%% 3
\item Calcule el rango y la inversa (en caso de que exista) de la siguiente matriz:
\[
    A = 
    \begin{pmatrix}
    1 & 0  & 1  & 1\\ 
    1 & 1  & -1 & 2\\ 
    2 & 0  & 1  & 0\\ 
    0 & -1 & 1  & -3
    \end{pmatrix}
\]

\begin{sol}
Por medio de operaciones elementales en los renglones veremos si podemos obtener su inversa. Observando que
el renglón 1, 2 y 4 son combinación lineal del renglón 4. Verficándolo explícitamente,

\[
    \begin{pmatrix}
        1 & 0  & 1  & 1\\ 
        1 & 1  & -1 & 2\\ 
        2 & 0  & 1  & 0\\ 
        0 & -1 & 1  & -3
    \end{pmatrix}
    \quad\longrightarrow\quad
    \begin{pmatrix}
        1 & 0  & 1  & 1\\ 
        0 & 1  & -2 & 2\\ 
        2 & 0  & 1  & 0\\
        0 & -1 & 1  & -3
    \end{pmatrix}
    \quad\longrightarrow\quad
    \begin{pmatrix}
        1 & 0  & 1  & 1\\ 
        1 & 1  & -1 & 3\\ 
        2 & 0  & 1  & 0\\ 
        0 & -1 & 1  & -3
    \end{pmatrix}
    \quad\longrightarrow\quad
\]
\[
    \begin{pmatrix}
        1 & 0  & 1  & 1\\ 
        1 & 1  & -1 & 3\\ 
        0 & -2 & -2 & -6\\ 
        0 & -1 & 1  & -3
    \end{pmatrix}
    \quad\longrightarrow\quad
    \begin{pmatrix}
        1 & 0  & 1  & 1\\ 
        1 & 1  & -1 & 3\\ 
        0 & -2 & -2 & -6\\ 
        0 & 0 & 0  & 0
    \end{pmatrix}
\]
Usando el hecho que el rango es igual al número máximo de renglones linealmente independientes, siendo este
$\rank(A) = 3$. Por tanto, la matriz $A$ no tiene inversa.
\end{sol}

%%%%%%%% 4
\item Sea $T : \Po_2(\R) \to \Po_2(\R)$ dada por $T(f(x)) = f(x) + f'(x) + f''(x)$:
\begin{enumerate}[label=(\alph*)]
    \item Usando el concepto del rango de una matriz demuestre que la transformación $T$ es invertible.
    \begin{sol}
    Tomemos $\beta$ la base estandar ordenada de $\Po_2(\R)$. Teniendo así que:
    \begin{align*}
        T(1) &= 1 \cdot 1 + 0 \cdot x + 0 \cdot x^2\\
        T(x) &= 1 \cdot 1 + 1 \cdot x + 0 \cdot x^2\\
        T(x^2) &= 2 \cdot 1 + 2 \cdot x + 1 \cdot x^2
    \end{align*}
    De ahí que,
    \[
        [T]_\beta =
        \begin{pmatrix}
            1 & 1 & 2 \\ 
            0 & 1 & 2 \\ 
            0 & 0 & 1
        \end{pmatrix}
    \]
    
    Como las columnas son todos linealmente independientes se sigue que,
    \[
        \rank([T]_\beta) = \dim\left(
            \Span\left(
                \left\{
                    \begin{pmatrix}
                        1\\ 
                        0\\ 
                        0
                    \end{pmatrix},
                    \begin{pmatrix}
                        1\\ 
                        1\\ 
                        0
                    \end{pmatrix},
                    \begin{pmatrix}
                        2\\ 
                        2\\ 
                        1
                    \end{pmatrix}
                \right\}
            \right)
            \right) = 3
            \]
            
            Lo que implica que $T$ es invertible.
    \end{sol}

    \item Usando la matriz inversa obtenga la transformación inversa de $T$.
    \begin{sol}
    Se tratará por medio de operaciones elementales de renglones transformar $([T]_\beta | I_3)$ en una
    matriz de la forma $(I_3 | [T]_\beta^{-1})$.

    \[
        \left(
        \begin{array}{ccc|ccc}
            1 & 2 & 2 & 1 & 0 & 0 \\
            0 & 1 & 2 & 0 & 1 & 0 \\
            0 & 0 & 1 & 0 & 0 & 1
        \end{array}
        \right)
        \quad\longrightarrow\quad
        \left(
        \begin{array}{ccc|ccc}
            1 & 0 & 0 & 1 & -1 & 0 \\
            0 & 1 & 0 & 0 & 1 & -2 \\
            0 & 0 & 1 & 0 & 0 & 1
        \end{array}
        \right)
    \]

    Se sigue que,
    \[
        ([T]_\beta)^{-1} =
        \begin{pmatrix}
            1 & -1 & 0\\ 
            0 & 1 & -2\\ 
            0 & 0 & 1
        \end{pmatrix}
    \]

    Además se tiene que,
    \begin{align*}
        [T^{-1}(a_0 + a_1x + a_2 x^2)]_\beta
            &= \begin{pmatrix}
                    1 & -1 & 0\\ 
                    0 & 1 & -2\\ 
                    0 & 0 & 1
                \end{pmatrix}
                \begin{pmatrix}
                    a_0\\ 
                    a_1\\
                    a_2
                \end{pmatrix}\\
            &= \begin{pmatrix}
                    a_0 - a_1\\
                    a_1 - 2a_2\\
                    a_2
                \end{pmatrix}
    \end{align*}

    Por tanto
    \[
        T^{-1}(a_0 + a_1x + a_2 x^2) = (a_0 - a_1) + (a_1 - 2a_2)x + a_2 x^2
    \]
    \end{sol}

    \item Verifique que la transformación obtenida en el paso anterior es efectivamente la transformación inversa.
    \begin{sol}
        Es invertible porque
        \[
            [T]_\beta ([T]_\beta)^{-1} = ([T]_\beta)^{-1} [T]_\beta= I_3
        \]
    \end{sol}
\end{enumerate}

%%%%%%%% 5
\item Para cada uno de los siguientes sistemas de ecuaciones encuentre la base y la dimensión del subespacio
de soluciones para el sistema homogéneo y luego encuentre todas las soluciones para el sistema de
ecuaciones no homogéneo:
\begin{enumerate}[label=(\alph*)]
    \item
    \begin{equation*}
        \begin{split}
            x_1 + 3x_2 &= 0\\
            2x_1 + 6x_2 &= 0
        \end{split} 
        \qquad\qquad
        \begin{split}
            x_1 + 3x_2 &= 5\\
            2x_1 + 6x_2 &= 10
        \end{split}
    \end{equation*}
    
    \begin{sol}
    Consideremos el sistema homogéneo de ecuaciones lineales y sea $A$ su forma de matriz,
    \[
        A =
        \begin{pmatrix}
            1 & 3\\ 
            4 & 6
        \end{pmatrix}
    \]
    Se observar fácilmente que sus renglones son linealmente dependiente porque el segundo renglón
    se puede expresar como combinación lineal del primero,
    \[
        \begin{pmatrix}
            1 & 3\\ 
            4 & 6
        \end{pmatrix}
        \quad\longrightarrow\quad
        \begin{pmatrix}
            1 & 3\\ 
            0 & 0
        \end{pmatrix}
    \]
    Tendiendo así que $\rank(A) = 1$.

    Usando un teorema\footnote{
        \textbf{Teorema 3.8} \emph{Sea $Ax = 0$ sea un sistema homogéneo de $m$ ecuaciones lineales con $n$ incógnitas
        sobre un campo $F$. Sea $K$ que denota el conjunto de otdas las soluciones de $Ax = 0$. Entonces $K = N(L_A)$;
        de ahí que $K$ es el subespacio de $F^n$ de dimensión $n -  \rank(L_A) = n - \rank(A)$.}
    } podemos concluir que el subespacio de soluciones $K$ su $\dim(K) = 2 - 1 = 1$.

    Consideremos a $
    \begin{pmatrix}
        -1\\
        3
    \end{pmatrix}$ una solución del sistema de ecuaciones, siendo esta una base para $K$.

    De ahí que $K = \left\{
        t \cdot
        \begin{pmatrix}
            -1\\
            3
        \end{pmatrix} \: : \: t \in \R
    \right\}$ son las soluciones para el sistem a de ecuaciones homogéneo.

    \noindent \rule{\linewidth}{0.2mm}

    Para el sistema no homogéneo, consideromos a $\begin{pmatrix}
        5\\
        0
    \end{pmatrix}$ una solución particular pero por otro teorema\footnote{
        \textbf{Teorema 3.9} \emph{Sea $K$ el conjunto de soluciones de un sistema de ecuaciones lineales $Ax = b$},
        y sea $K_H$ una conjunto de solucinoes del sistema homogéneo correspondiente $Ax = 0$. Entonces cualquier
        solución $s$ a $Ax = b$ es de la forma
        $$ K = \{ s \} + K_H = \{ s + k \: : \: k \in K_H \}. $$
    }
    podemos concluir que el conjunto de soluciones es:
    \[
        K = \left\{
        \begin{pmatrix}
            5\\
            0
        \end{pmatrix} +
        t \cdot
        \begin{pmatrix}
            -1\\
            3
        \end{pmatrix}
        \: : \: t \in \R
        \right\}
    \]

    \end{sol}

    \item
    \begin{equation*}
        \begin{split}
            2x_1 + x_2 - x_3 &= 0\\
            x_1 - x_2 + x_3 &= 0\\
            x_1 + 2x_2 - 2x_3 &= 0
        \end{split}
        \qquad\qquad
        \begin{split}
            2x_1 + x_2 - x_3 &= 5\\
            x_1 - x_2 + x_3 &= 1\\
            x_1 + 2x_2 - 2x_3 &= 4
        \end{split}
    \end{equation*}

    \begin{sol}
    Consideremos el sistema homogéneo de ecuaciones lineales y sea $A$ su forma de matriz,
    \[
        A =
        \begin{pmatrix}
            2 & 1  & -1\\
            1 & -1 & 1\\
            1 & 2  & -2
        \end{pmatrix}
    \]
    Se observar fácilmente que la segunda columna de la matriz es una combinación lineal de la tercera columna,
    \[
        \begin{pmatrix}
            2 & 1  & -1\\
            1 & -1 & 1\\
            1 & 2  & -2
        \end{pmatrix}
        \quad\longrightarrow\quad
        \begin{pmatrix}
            2 & 1  & 0\\
            1 & -1 & 0\\
            1 & 2  & 0
        \end{pmatrix}
    \]
    Tendiendo así que $\rank(A) = 2$.

    Concluyendo así que $\dim(K) = 3 -2 = 1$.

    Consideremos a $
    \begin{pmatrix}
        0\\
        1\\
        1
    \end{pmatrix}$ una solución del sistema de ecuaciones, siendo esta una base para $K$.

    De ahí que $K = \left\{
        t \cdot
        \begin{pmatrix}
            0\\
            1\\
            1
        \end{pmatrix} \: : \: t \in \R
    \right\}$ son las soluciones para el sistem a de ecuaciones homogéneo.

    \noindent \rule{\linewidth}{0.2mm}

    Para el sistema no homogéneo, consideromos a $\begin{pmatrix}
        2\\
        1\\
        0
    \end{pmatrix}$ una solución particular.
    
    Podemos concluir que el conjunto de soluciones es:
    \[
        K = \left\{
        \begin{pmatrix}
            2\\
            1\\
            0
        \end{pmatrix} +
        t \cdot
        \begin{pmatrix}
            0\\
            1\\
            0
        \end{pmatrix}
        \: : \: t \in \R
        \right\}
    \]

    \end{sol}
\end{enumerate}

\end{enumerate}


%%%%%%%%%%%%%%%%%%%%%%%%%%%%%%%%%%%%%%%%%%%%%%%%%%%%%%%%%%%%%%%%%%%%%%%%%%%%%%%%%%%%%%%%%
\end{document}