%%%
 %
 % Copyright (C) 2019 Ángel Iván Gladín García
 %
 % This program is free software: you can redistribute it and/or modify
 % it under the terms of the GNU General Public License as published by
 % the Free Software Foundation, either version 3 of the License, or
 % (at your option) any later version.
 %
 % This program is distributed in the hope that it will be useful,
 % but WITHOUT ANY WARRANTY; without even the implied warranty of
 % MERCHANTABILITY or FITNESS FOR A PARTICULAR PURPOSE.  See the
 % GNU General Public License for more details.
 %
 % You should have received a copy of the GNU General Public License
 % along with this program.  If not, see <http://www.gnu.org/licenses/>.
%%%

%%%%%%%%%%%%%%%%%%%%%%%%%%%%%%%%%%%%%%%%%%%%%%%%%%%%%%%%%%%%%%%%%%%%%%%%%%%%%%%%%%%%%%%%%
\documentclass[11pt,letterpaper]{article}
\usepackage[margin=.75in]{geometry}
\usepackage[utf8]{inputenc}
\usepackage[spanish]{babel}
\decimalpoint

\usepackage{listings}
\usepackage{color}
\usepackage{graphicx}
\usepackage{enumerate}
\usepackage{enumitem}
\usepackage{float}

\usepackage{longtable}
\usepackage{hyperref}
\usepackage{commath}

\usepackage{bbm}
\usepackage{dsfont}
\usepackage{mathrsfs}
\usepackage{amsmath,amsthm,amssymb}
\usepackage{mathtools}
\usepackage{longtable}

\usepackage{tikz}
\usetikzlibrary{trees}
\usepackage{verbatim}

\usepackage{systeme}


%%%%%%%%%%%%%%%%%%%%%%%%%%%%%%%%%%%%%%%%%%%%%%%%%%%%%%%%%%%%%%%%%%%%%%%%%%%%%%%%%%%%%%%%%%%%%%%%5

\usepackage{import}

\usepackage[utf8]{inputenc}
%%%%%%%%%%%%%%%%%%%%%%%%%%%%%%%%%%%%%%%%%%%%%%%%%%%%%%%%%%%%%%%%%%%%%%%%%%%%%%%%%%%%%%%%%


%%%%%%%%%%%%%%%%%%%%%%%%%%%%%%%%%%%%%%%%%%%%%%%%%%%%%%%%%%%%%%%%%%%%%%%%%%%%%%%%%%%%%%%%%
\newcommand{\Z}{\mathbb{Z}}
\newcommand{\N}{\mathbb{N}}
\newcommand{\Q}{\mathbb{Q}}
\newcommand{\R}{\mathbb{R}}
\newcommand{\Pro}{\mathds{P}}
\newcommand{\Oh}{\mathcal{O}} %% Notacion "O"
\newcommand{\lra}{\longrightarrow}
\newcommand{\ra}{\rightarrow}
\newcommand{\ord}{\text{ord}}
\newcommand{\sol}{\textbf{\underline{Solución}: }} %% Solucion
\newcommand{\af}{\textbf{\underline{Afirmación}: }}
\newcommand{\cej}{\textbf{\underline{Contraejemplo}: }}

\newcommand{\verd}{\textbf{\underline{Verdadero} }}
\newcommand{\fals}{\textbf{\underline{Falso} }}

%%%%%%%%%%%%%%%%%%%%%%%%%%%%%%%%%%%%%%%%%%%%%%%%%%%%%%%%%%%%%%%%%%%%%%%%%%%%%%%%%%%%%%%%%

\begin{document}

%%%%%%%%%%%%%%%%%%%%%%%%%%%%%%%%%%%%%%%%%%%%%%%%%%%%%%%%%%%%%%%%%%%%%%%%%%%%%%%%%%%%%%%%%
\title{
        Universidad Nacional Autónoma de México\\
        Facultad de Ciencias\\
        Álgebra Lineal I\\
    \vspace{.5cm}
    \large
        \textbf{Tarea 1}
}
\author{
    Ángel Iván Gladín García\\
    No. cuenta: 313112470\\
    \texttt{angelgladin@ciencias.unam.mx}
}
\date{5 de Marzo 2020}
\maketitle
%%%%%%%%%%%%%%%%%%%%%%%%%%%%%%%%%%%%%%%%%%%%%%%%%%%%%%%%%%%%%%%%%%%%%%%%%%%%%%%%%%%%%%%%%

%%%%%%%%%%%%%%%%%%%%%%%%%%%%%%%%%%%%%%%%%%%%%%%%%%%%%%%%%%%%%%%%%%%%%%%%%%%%%%%%%%%%%%%%%
\newtheorem{theorem}{Teorema}
\newtheorem{example}{Ejemplo}
\newtheorem{corollary}{Corolario}
\newtheorem{lemma}{Lemma}
\newtheorem{definition}{Definicion}
\newtheorem{prop}{Proposicion}
%%%%%%%%%%%%%%%%%%%%%%%%%%%%%%%%%%%%%%%%%%%%%%%%%%%%%%%%%%%%%%%%%%%%%%%%%%%%%%%%%%%%%%%%%

%%%%%%%%%%%%%%%%%%%%%%%%%%%%%%%%%%%%%%%%%%%%%%%%%%%%%%%%%%%%%%%%%%%%%%%%%%%%%%%%%%%%%%%%%
\begin{enumerate}
\item ¿Es verdadera o falsa cada una de las siguientes afirmaciones acerca de cualquier espacio vectorial $V$?
Justifica tu respuesta.
\begin{enumerate}[label=(\alph*)]
    \item Si para algún vector $x \in V$ se cumple que $ax = bx$, entonces se tiene que $a = b$.
    
    \fals. Tomemos $x \in V$ donde $x = 0$ y $a,b \in F$ arbitrarios donde $a = 1$ y $b = 2$. Se tiene que
    $0 = 1 \cdot 0 = 2 \cdot 0 = 0$ pero $a \neq b$.

    \item Si para todo escalar $a$ se cumple que $ax = ay$ con $x,y \in V$, entonces $x = y$.
    
    \fals. Similar al inciso anterior. Tomando $a \in F$ tal que $a = 0$ y $x,y \in V$ tal que
    $x \neq y$, siempre se cumple que $ax = ay$ pero se tomaron $x, y$ distintos, por lo tanto
    es falso.
\end{enumerate}

\item Sea $V$ un espacio vectorial sobre un campo $K$. Demuestre que:
\begin{enumerate}[label=(\alph*)]
    \item El neutro aditivo en $V$ es único.
    \begin{proof}
        Supongamos que hay dos vectores $0$ y $0'$. Usando VS3\footnote{Existe un elemento en $V$ denotado como $0$
        tal que $x+0=0$ para cada $x \in V$.} tenemos entonces que para cualquier vector $x$ se tiene que
        $x + 0 = x = x + 0'$, pero ahora lo que tenemos que dementrar es que $0 = 0'$.
        \begin{proof} (La Ley de la cancelación para la suma de vectores)

            Si $x,y$ y $z$ son vectores en un espacio vectorial $V$ tal que $x + z = y + z$, entonces
            $x = y$.
            
            Existe un vector $v \in V$ tal que $z + v= 0$ por VS4\footnote{Para cada elemento $x \in V$ existe
            un elemento $y$ tal que $x+y=0$.}. Por tanto
            \begin{align*}
                x &= x + 0 = x + (z + v) = (x + z) + v\\
                  &= (y + z) + v = y + (z + v) = y + 0 = y
            \end{align*}
        \end{proof}
        Entonces se puede concluir que $0 = 0'$, i.e., el netro aditivo es único.
    \end{proof}

    \item Para cada vector $x \in V$ su invierso aditivo es único.
    \begin{proof}
        Dado el vector $x \in V$ y sean $y, y' \in V$ tal que satisfacen VS4. Entonces
        $x + y = 0 = x + y'$, pero por el inciso anterior se demostró  ``La Ley de la cancelación para
        la suma de vectores'', entonces ``cancelando'' a $x$ se concluye que $y = y'$.
    \end{proof}
\end{enumerate}

\item Sean $V$ un espacio vectorial, $x,y \in V$ y $a,b \in K$. Demuestre que
$(a + b)(x + y) = ax + ay + bx + by$.
\begin{proof}
    Usando VS7\footnote{Para cada elemento $a \in F$ y cadapar de elementos $x,y \in V$, 
    $a(x+y)=ax + ay$.} y VS8\footnote{Para cada par de ementos $a,b \in F$ y cada elemento
    $x \in V$, $(a+b)x=ax+bx$.} se tiene que,
    \begin{align*}
        (a + b)(x + y)
            &= (a + b)x + (a + b)y && \text{Por VS7}\\
            &= ax + ay + bx + by && \text{Por VS8}
    \end{align*}
\end{proof}

\item Demuestre que el conjunto de las funciones pares en $\R$:
\[
    \mathscr{E}(\R, \R) = \{ f: \R \to \R \  \mid \  f(x) = f(-x), \forall x \in \R \}
\]
junto con la suma de funciones y el producto por escalar usuales forman un espacio vectorial.
\begin{proof}
    Lo primero que se debe verificar es que si $f, g \in \mathscr{E}$ y $t \in \R$ entonces
    $f+g, cf \in \mathscr{E}$ de modod que la suma y multiplicar por un escalar están bien
    definidas en $\mathscr{E}$ sobre $\R$. Por tanto si $f,g \in \mathscr{E}$ y $c \in \R$ entonces,
    \begin{align*}    
        (f+g)(t)=f(-t)+g(-t)=f(t)+g(t)=(f+g)(t),\\
        (cf)(-t)=c[f(-t)]=c[f(t)]=(cf)(t)
    \end{align*}
    para toda $t \in \R$, llo que implica que $f+g$ y $cf$ son funciones pares en $\mathscr{E}$.
    Por tanto están bien definidas.

    Para provar que es un espacio vectorial debemos de mostrar que se cumplen VS1 - VS8.
    \begin{itemize}
        \item \textbf{VS1}. Si $f,g \in \mathscr{E}$ y para toda $t \in \R$,
        entonces la suma es conmutativa, teniendo así que,
        \[ (f+g)(t) = f(t)+g(t)=g(t)+f(t)=(g+f)(t) \]
        lo que implica que $f+g=g+f$.
        
        \item \textbf{VS2} Si $f,g,h \in \mathscr{E}$, entonces la suma es asosiativa, teniendo
        \begin{align*}
            [(f+g)+h](t) 
                &= (f+g)(t) + h(t) = (f(t) + g(t)) + h(t)\\
                &= f(t) + (g(t) + h(t)) = [f + (g+h)](t)
        \end{align*}
        para toda $t \in \R$, teniendo así que $(f+g)+h=f+(h+g)$.
        
        \item \textbf{VS3} Sea $0$ la función constante $0: \R \to \R$ tal que $0 \in \mathscr{E}$
        definida como $0(t)=0$ para toda $t \in \R$. Entonces $0(-t)= 0 = 0(t)$.
        Sea $f \in \mathscr{E}$ entonces se cumple que,
        \[ (f+0)(t) = f(t) + 0(t) = f(t) + 0 = f(t) \]
        Teniendo así qiie $f + 0 = f$.

        \item \textbf{VS4} Sean $f, g \in \mathscr{E}$ donde $g(t)=-f(t)$ para toda $t \in \R$, como
        $f \in \mathscr{E}$ y $g(-t)=-f(-t)=-f(t)$ y así $f + g = 0$ porque
        \[ (f+g)(t) = f(t) + g(t) = f(t)-f(t) = 0 = 0t  \]

        \item \textbf{VS5} Sea $f \in \mathscr{E}$, se tiene que $1f = f$ y para toda $t \in \R$, dado que
        $(1f)(t)= 1[f(t)] = f(t)$.
        
        \item \textbf{VS6} Para cada $a,b \in \R$ y $f \in \mathscr{R}$ y para toda $t \in \R$, se tiene,
        $(ab)f=a(bf)$ porque
        \[ [(ab)f](t) = (ab)[f(t)] = a[bf(t)] = a[(bf)(t)] \]
        
        \item \textbf{VS7} Para cada $a \in \R$ y $f, g \in \mathscr{E}$ y para toda $t \in \R$,
        se tiene $a(f+g)=af+ag$ porque
        \begin{align*}
            [a (f+g)](t) 
                &= a [(f + g) (t)] = a (f (t) + g (t)) = a[f(t)] + a[g(t)]\\
                &= (af)(t) + (ag)(t) = (af + ag)(t)
        \end{align*}

        \item \textbf{VS8} Para cada $a,b \in \R$ y $f \in \mathscr{E}$ y para toda $t \in \R$,
        se tiene $(a+b)f=af+bf$ porque
        
    \end{itemize}
\end{proof}

¿Qué se puede decir del conjunto de las funciones impares, forman o no un espacio vectorial? Justifique su respuesta.

\verd Sí forman un espacio vectorial, denotando así
\[
    \mathscr{E}'(\R, \R) = \{ f: \R \to \R \  \mid \  f(-x) = -f(x), \forall x \in \R \}
\]
\begin{proof}
    Sea $\mathscr{E}' \subset V$ donde $V$ está definido como el espacio vectorial de todos los funciones
    evaluadas en valores reales en la línea real. Como $V$ es un espacio vectorial basta con probar que
    $\mathscr{E}'$ es un subespacio de $V$.
    
    La función $0$ es impar, $0_V(-x) = -0_V(x)$ para todo $x \in \R$. Ahora sea $c \in \R$ y $f, g
    \in \mathscr{E}'$ entonces $(cf+g)(-x) = cf(-x) + g(-x) = -cf(x) - g(x) = -(cf+g)(x)$
    y por tanto $cf + g \in \mathscr{E}'$. Como $\mathscr{E}'$ es
    un subespacio de sigue que es un espacio vectorial.
\end{proof}

\item Considere el espacio vectorial $\mathscr{F}(S,\R)$ y $f,g,h \in \mathscr{F}(S,\R)$ definidos como
$f(s)=2s+1$, $g(s)=1+4s-2s^2$ y $h(t)=5^t+1$.
\begin{enumerate}[label=(\alph*)]
    \item Demuestre que las tres funciones son diferentes cuando $S = \R$.
    \begin{proof}
        Basta con tomar valores diferentes a los valores de $S$.
    \end{proof}

    \item Demuestre que $f=g$ y $f+g=h$ cuando $S=\{0,1\}$
    \begin{proof}
        \begin{enumerate}
            \item $f = g$

            Se debe mostrar que $f(t) = g(t)$ para $t \in S$. Esto se cumple porque 
            $f(0) = 2(0)+1 = 1 = 1 + 4(0) - 2(0)^2 = g(0)$ y
            $f(1) = 2(1)+1 = 3 = 1 + 4(1) - 2(1)^2 = g(1)$.

            \item $f + g = h$

            Se debe mostrar que $f(t) + g(t) = h(t)$ para $t \in S$. Igual se cumple porque
            $f(1) + g(1) = 3 + 3 = 6 = 5^{(1)} + 1 = h(1)$ y
            $f(0) + g(0) = 1 + 1 = 2 = 5^{(0)} + 1 = h(0)$.
        \end{enumerate}
    \end{proof}
\end{enumerate}

\item Demuestre o de un contraejemplo para las siguinetes afirmaciones:
\begin{enumerate}[label=(\alph*)]
    \item La intersección de dos subespacioes vectoriales es un subespacio vectorial
    \begin{proof}
        TODO
    \end{proof}

    \item La suma de dos subespacios vectoriales forma un subespacio vectorial.
    
    TODO

    \item La suma de dos subconjuntos de un espacio vectorial siempre es un subespacio vectorial.
    
    TODO

    \item El conjunto $\{f \in \mathscr{E} \ \mid \ f (1) = 1 \}$ es un subespacio vectorial de $\mathscr{E}$.
    
    TODO

\end{enumerate}

\item ¿Es posible expresar al vector $(5,1,-5)$ como combinación lineal de elementos en\\
$S = \{(1, -2, -3), (-2, 3, -4)\}$.

Para ver si es posible expresarlo como combinación lineal de elemntos de $S$ basta
encontrar un $\alpha, \beta \in \R$ tal que:
\[ (5,1,-5) = \alpha(1,-2,-2) + \beta(-2,3,4) \]
Teniendo así que $(5,1,-5) = (\alpha - 2\beta, -2\alpha + 3\beta, -\alpha-4\beta)$.

Resolviendo el siguiente sistema de ecuaciones,
\[
\systeme{5=\alpha - 2\beta,
        1= -2\alpha + 3\beta,
        $-5$ =-\alpha-4\beta}
\]

Multiplicando la primera ecuación por dos y sumándola a la segunda se tiene,
\[
\systeme{5=\alpha - 2\beta,
        11= -\beta,
        5 =-\alpha-4\beta}
\]
Multiplinado la primera ecuación por 3 y sumándola a la tercera se tiene,
\[
\systeme{5=\alpha - 2\beta,
        11= -\beta,
        10 =-10\beta}
\]
Pero no hay un valor que pueda tomar $\beta$ para que satisfaga el sistema de ecuaciones,
por lo tanto el vector $(5,1,-5)$ \textbf{no puede ser expresado} como combinación lineal
de los elementos de $S$.

\item ¿Es el vector $6x^3 - 3x^2 + x + 2$ una combinación lineal de los vectores
$x^3 -x^2 +2x + 3$ y $2x^3 -3x+3$. (Recuerde que los polinomios $\mathds{P}[x]$ forman un
espacio vectorial sobre $\R$ con la suma y producto por un escalar usuales).

Para provar si el primer polinomio se puede expresar como combinación lineal de los otros dos
debemos de encontrar $\alpha, \beta \in \R$ tal que,
\[ 6x^3 - 3x^2 + x + 2 = \alpha(^3 -x^2 +2x + 3) + \beta(2x^3 -3x+3) \]
Teniendo así que,
\begin{align*}
    6x^3 - 3x^2 + x + 2
        &= \alpha x^3 - \alpha x^2 + 2 \alpha x + 3 \alpha + 2 \beta x^3 - 3 \beta x + \beta\\
        &= (\alpha + 2 \beta)x^3 - \alpha x^2 + (2\alpha - 3 \beta)x + 3 \alpha + \beta
\end{align*}
Resolviendo el siguiente sistema de ecuaciones (agrupando con resecto a cada coeficiente del polinomio)
\[
    \systeme{6 = \alpha + 2 \beta,
            $-3$ = -\alpha,
            1 = 2\alpha - 3\beta,
            2 = 3\alpha + \beta}
\]
Sumando la primera ecuación a la segunda se tiene,
\[
    \systeme{6 = \alpha + 2 \beta,
            3 = 2 \beta,
            1 = 2\alpha - 3\beta,
            2 = 3\alpha + \beta}
\]
Multiplicando por $(-2)$ la primera ecuación y sumándosela a la tercera se tiene,
\[
    \systeme{6 = \alpha + 2 \beta,
            3 = 2 \beta,
            $-11$ = -7\beta,
            2 = 3\alpha + \beta}
\]
Multiplicando por $(-3)$ la primera ecuación y sumándosela a la tercera se tiene,
\[
    \systeme{6 = \alpha + 2 \beta,
            3 = 2 \beta,
            $-11$ = -7\beta,
            $-16$ = -5 \beta}
\]

Como no hay un valor $\beta$ que satisfaga la ecuación se concluye que el vector
$6x^3 - 3x^2 + x + 2$ \textbf{no puede ser expresado como combinación lineal de los vectores}
$x^3 -x^2 +2x + 3$ y $2x^3 -3x+3$.


\item Considera el espacio vectorial de las matrices simétricas de $2 \times 2$.
Exhiba una base para este espacio vectorial y justifique su respuesta.

Expresando la base canónica se tiene entonces,
\[
\begin{bmatrix}
    1  &  0      \\
    0  &  0      
\end{bmatrix}
, 
\begin{bmatrix}
    0  &  1      \\
    0  &  0      
\end{bmatrix}
, 
\begin{bmatrix}
    0  &  0      \\
    1  &  0      
\end{bmatrix}
, 
\begin{bmatrix}
    0  &  0      \\
    0  &  1      
\end{bmatrix} 
\]
Sea $
    \begin{bmatrix}
    a  &  b      \\
    c  &  d      
    \end{bmatrix} \in M_{2 \times 2}$ entonces se puede expresar como,
    \[
        \begin{bmatrix}
            a  &  b      \\
            c  &  d      
        \end{bmatrix}
        =
        a\begin{bmatrix}
            1  &  0      \\
            0  &  0      
        \end{bmatrix}
        +
        b\begin{bmatrix}
            0  &  1      \\
            0  &  0      
        \end{bmatrix}
        + 
        c\begin{bmatrix}
            0  &  0      \\
            1  &  0      
        \end{bmatrix}
        + 
        d\begin{bmatrix}
            0  &  0      \\
            0  &  1      
        \end{bmatrix} 
        \]


%%%%%%% 10
\item Sean $V$ un espacio vectorial y $S,T \subset V$ tales que $S \subseteq T$. Si $S$ es
linealmente independiente ¿es $T$ linealmente independiente?

\fals

Lo que sí se cumple es: \textbf{Si $T$ es linealmente independiente, entonces $S$ el
linealmente independiente}.

\begin{proof}
    Sean $S = \{ v_1, v_2, \ldots, v_j \}$ y $T = \{ v_1, v_2, \ldots, v_j, v_{j+1}, \ldots, v_k \}$.
    Como $T$ es linealmente independiente, entonces $\sum_{i=1}^{k} a_i x_i = 0$, lo que por definición
    de independia lineal pasa si y sólo si $a_i = 0$ con $1 \leq i \leq k$. Como
    $a_{j+1} = a_{j+1} = \ldots = a_{k} = 0$, entonces $\sum_{i=j+1}^{k} a_i x_i = 0$,
    teniendo así que $\sum_{i=1}^{j} a_i x_i = 0$ si y sólo si $a_{1} = a_{2} = \ldots = a_{k} = 0$.
    Por tanto $S$ es linealmente independiente.
\end{proof}

%%%%%%% 11
\item Sean $V$ un espacio vectorial y $S,T \subset V$ tales que $S \subseteq T$. Si $T$
es linealmente dependiente ¿es $S$ linealmente dependiente?

\fals

Lo que sí se cumple es: \textbf{Si $S$ es linealmente dependiente, entonces $T$ el
linealmente dependiente}.

\begin{proof}
    Sean $S = \{ v_1, v_2, \ldots, v_j \}$ y $T = \{ v_1, v_2, \ldots, v_j, v_{j+1}, \ldots, v_k \}$.
    Como $S$ es linealmente dependiente, entonces por definición se tiene que existe una constante
    $a_1, a_2, \ldots, a_j$ no todos cero tal que $\sum_{i=1}^{j} a_i v_i = 0$. Entonces tomemos
    $a_{j+1} = a_{j+1} = \ldots = a_{k} = 0$, se sigue que $\sum_{i=i}^{k} a_i v_i = 0$ y al menos un 
    $a_i \neq 0$ para $1 \leq i \leq k$. Por tanto $S$ es linealmente dependiente.
\end{proof}

\item Sean $V$ un espacio vectorial y $x, y \in V$ . Demuestre que el conjunto $\{x, y\}$
es un conjunto linealmente dependiente si y sólo si uno es múltiplo del otro.

\begin{proof}
    \hfill
    \begin{itemize}
        \item[$\Longrightarrow)$] Como $\{x, y\}$ son linealmente independientes. Entonces
        $\alpha x + \beta y = 0$ cuando $\alpha, \beta$ no son 0. Supongamos que $\alpha \neq 0$,
        entonces $x = - \frac{\beta}{\alpha}y$ y por lo tanto $x$ es múltiplo de $y$.

        \item[$\Longleftarrow)$] Como $\{x, y\}$ son múltiplos, i.e., $x = \alpha y$, entonces
        donde $x = 1x$ y como está siendo multiplicado por 1, entonces puede que $\alpha$ sea
        o no 0, pero como al menos un coeficiente no es cero, entonces cumple con la definición
        de ser linealmente dependiente.
    \end{itemize}
\end{proof}


%%%%%%% 13
\item Considere espacios vectoriales $V$ y $W$ sobre un campo $K$ y una transformación lineal
$T : V \to W$ . Demuestre que:
\begin{enumerate}[label=(\alph*)]
    \item $T(0_V) = 0_W$.
    \begin{proof}
        \[ T(0 \cdot 0_V) = 0 T(0_V) = 0_W \]
    \end{proof}
    
    \item $T$ es una transformación lineal si y sólo si $T (ax + y) = aT (x) + T(y)$ para
    cualesquiera $x,y \in V$ y $a \in K$.
    \begin{proof}
        \hfill
        \begin{itemize}
            \item[$\Longrightarrow)$] Como $T$ es lineal y sean $x,y \in V$ y $a \in K$, entonces
            $T(ax + y) = aT(x) + T(y)$. Por pura definición.
    
            \item[$\Longleftarrow)$] Como $T (ax + y) = aT (x) + T(y)$ para
            cualesquiera $x,y \in V$ y $a \in K$. Entonces,

            \begin{itemize}
                \item $T(x + y) = T(1x + y) = 1T(x) + T(y) = T(x) + T(y)$.
                \item $T(cx) = T(cx + 0_V) = cT(x) + T(0_V) = cT(x) + 0_W = cT(x)$.
            \end{itemize}
        \end{itemize}
    \end{proof}

    \item Para cualesquiera $x, y \in V$ se cumple que $T(x-y) = T(x)-T(y)$.
    \begin{proof}
        \[ T(x-y) = T(x + (-y)) = T(x) + T(-y) = T(x) - T(y) \]
    \end{proof}
    
    \item $T$ es una transformación lineal si y sólo si
    $T \left( \sum_{i=1}^{n} a_ix_i \right) = \sum_{i=1}^{n}a_i T(x_i)$, para cualesquiera
    $x_1, \ldots, x_n \in V$ y $a_1, \ldots \in K$.
    \begin{proof}q
        \begin{align*}
            T \left( \sum_{i=1}^{n} a_ix_i \right)
                &= T(a_1 x_1) + T \left( \sum_{i=1}^{n-1} a_ix_i \right)\\
                &= T(a_1 x_1) + T(a_2 x_2) + \ldots + T(a_n x_n)\\
                &= \sum_{i=1}^{n} T(a_i x_i)\\
                &= \sum_{i=1}^{n} a_i T(x_i)
        \end{align*}
    \end{proof}
\end{enumerate}

\end{enumerate}
\begin{figure}[H]
    \centering
    \includegraphics[scale=0.5]{gato.jpg}
    \caption{Solo me faltó la 6 :(}
\end{figure}

%%%%%%%%%%%%%%%%%%%%%%%%%%%%%%%%%%%%%%%%%%%%%%%%%%%%%%%%%%%%%%%%%%%%%%%%%%%%%%%%%%%%%%%%%


%%%%%%%%%%%%%%%%%%%%%%%%%%%%%%%%%%%%%%%%%%%%%%%%%%%%%%%%%%%%%%%%%%%%%%%%%%%%%%%%%%%%%%%%%

%%%%%%%%%%%%%%%%%%%%%%%%%%%%%%%%%%%%%%%%%%%%%%%%%%%%%%%%%%%%%%%%%%%%%%%%%%%%%%%%%%%%%%%%%

\end{document}