%%%
 %
 % Copyright (C) 2019 Ángel Iván Gladín García
 %
 % This program is free software: you can redistribute it and/or modify
 % it under the terms of the GNU General Public License as published by
 % the Free Software Foundation, either version 3 of the License, or
 % (at your option) any later version.
 %
 % This program is distributed in the hope that it will be useful,
 % but WITHOUT ANY WARRANTY; without even the implied warranty of
 % MERCHANTABILITY or FITNESS FOR A PARTICULAR PURPOSE.  See the
 % GNU General Public License for more details.
 %
 % You should have received a copy of the GNU General Public License
 % along with this program.  If not, see <http://www.gnu.org/licenses/>.
%%%

%%%%%%%%%%%%%%%%%%%%%%%%%%%%%%%%%%%%%%%%%%%%%%%%%%%%%%%%%%%%%%%%%%%%%%%%%%%%%%%%%%%%%%%%%
\documentclass[11pt,letterpaper]{article}
\usepackage[margin=.75in]{geometry}
\usepackage[utf8]{inputenc}
\usepackage[spanish]{babel}
\decimalpoint

\usepackage{listings}
\usepackage{color}
\usepackage{graphicx}
\usepackage{enumerate}
\usepackage{enumitem}
\usepackage{float}

\usepackage{longtable}
\usepackage{hyperref}
\usepackage{commath}

\usepackage{bbm}
\usepackage{dsfont}
\usepackage{mathrsfs}
\usepackage{amsmath,amsthm,amssymb}
\usepackage{mathtools}
\usepackage{longtable}

\usepackage{tikz}
\usetikzlibrary{trees}
\usepackage{verbatim}

\usepackage{systeme}


%%%%%%%%%%%%%%%%%%%%%%%%%%%%%%%%%%%%%%%%%%%%%%%%%%%%%%%%%%%%%%%%%%%%%%%%%%%%%%%%%%%%%%%%%%%%%%%%5

\usepackage{import}

\usepackage[utf8]{inputenc}
%%%%%%%%%%%%%%%%%%%%%%%%%%%%%%%%%%%%%%%%%%%%%%%%%%%%%%%%%%%%%%%%%%%%%%%%%%%%%%%%%%%%%%%%%


%%%%%%%%%%%%%%%%%%%%%%%%%%%%%%%%%%%%%%%%%%%%%%%%%%%%%%%%%%%%%%%%%%%%%%%%%%%%%%%%%%%%%%%%%
\newcommand{\Z}{\mathbb{Z}}
\newcommand{\N}{\mathbb{N}}
\newcommand{\Q}{\mathbb{Q}}
\newcommand{\R}{\mathbb{R}}
\newcommand{\Pro}{\mathds{P}}
\newcommand{\Oh}{\mathcal{O}} %% Notacion "O"
\newcommand{\lra}{\longrightarrow}
\newcommand{\ra}{\rightarrow}
\newcommand{\ord}{\text{ord}}
\newcommand{\sol}{\textbf{\underline{Solución}: }} %% Solucion
\newcommand{\af}{\textbf{\underline{Afirmación}: }}
\newcommand{\cej}{\textbf{\underline{Contraejemplo}: }}

\newcommand{\verd}{\textbf{\underline{Verdadero} }}
\newcommand{\fals}{\textbf{\underline{Falso} }}

%%%%%%%%%%%%%%%%%%%%%%%%%%%%%%%%%%%%%%%%%%%%%%%%%%%%%%%%%%%%%%%%%%%%%%%%%%%%%%%%%%%%%%%%%

\begin{document}

%%%%%%%%%%%%%%%%%%%%%%%%%%%%%%%%%%%%%%%%%%%%%%%%%%%%%%%%%%%%%%%%%%%%%%%%%%%%%%%%%%%%%%%%%
\title{
        Universidad Nacional Autónoma de México\\
        Facultad de Ciencias\\
        Álgebra Lineal I\\
    \vspace{.5cm}
    \large
        \textbf{Tarea-Examen 1}
}
\author{
    Ángel Iván Gladín García\\
    No. cuenta: 313112470\\
    \texttt{angelgladin@ciencias.unam.mx}
}
\date{17 de Marzo 2020}
\maketitle
%%%%%%%%%%%%%%%%%%%%%%%%%%%%%%%%%%%%%%%%%%%%%%%%%%%%%%%%%%%%%%%%%%%%%%%%%%%%%%%%%%%%%%%%%

%%%%%%%%%%%%%%%%%%%%%%%%%%%%%%%%%%%%%%%%%%%%%%%%%%%%%%%%%%%%%%%%%%%%%%%%%%%%%%%%%%%%%%%%%
\newtheorem{theorem}{Teorema}
\newtheorem{example}{Ejemplo}
\newtheorem{corollary}{Corolario}
\newtheorem{lemma}{Lemma}
\newtheorem{definition}{Definicion}
\newtheorem{prop}{Proposicion}
%%%%%%%%%%%%%%%%%%%%%%%%%%%%%%%%%%%%%%%%%%%%%%%%%%%%%%%%%%%%%%%%%%%%%%%%%%%%%%%%%%%%%%%%%

%%%%%%%%%%%%%%%%%%%%%%%%%%%%%%%%%%%%%%%%%%%%%%%%%%%%%%%%%%%%%%%%%%%%%%%%%%%%%%%%%%%%%%%%%
\begin{enumerate}

\item Para los siguientes ejercicios considere $V$ un espacio vectorial.

\begin{enumerate}[label=(\alph*)]
    \item Sean $W_1, W_2 \leqslant V$ de dimensiones $m$ y $n$, respectivamente, tales que $m \geq n$.
    Demuestre que $dim(W_1 \cap W_2) \leq n$ y $dim(W_1 + W_2) \leq m + n$.
    \begin{proof}
        Sea $W_1 \cap W_2 \subseteq W_2$ y como $dim(W_2) = n$, entonces se sigue la siguiente desigualdad,
        \[ dim(W_1 \cap W_2) \leq dim(W_2) \]
        \[ dim(W_1 \cap W_2) \leq n \]
        Por tanto $dim(W_1 \cap W_2) \leq n$.

        \begin{theorem}
            Sean $W_1, W_2 \in V$ entonces,
            \[ dim(W_1 + W_2) = dim(W_1) + dim(W_2) - dim(W_1 \cap W_2) \]
        \end{theorem}
        Usando el teorema previamente enunciado, se sigue que,
        \begin{align*}
            dim(W_1 + W_2)
                &= dim(W_1) + dim(W_2) - dim(W_1 \cap W_2)\\
                &= n + m - dim(W_1 \cap W_2)
        \end{align*}
        Como $dim(W_1 \cap W_2) \leq n \leq m + n$, entonces $dim(W_1 + W_2) \leq m + n$.
    \end{proof}

    \item Demostrar que si $\{ v_1, v_2, v_3 \}$ es una base de $V$ entonces
    $\{ v_1 + v_2 + v_3, v_2 + v_3, v_3 \}$ también es una base de $V$.
    \begin{proof}
        Sea $x \in V$, tal que $a(v_1 + v_2 + v_3) + b(v_2 + v_3) + cv_3 = x$, entonces
        \[ av_1 + av_2 + av_3 + bv_2 + bv_3 + cv_3 = x \quad \iff \quad
            av_1 + (a+b)v_2 + (a+b+c)v_3 = x \]
        Como $v_1, v_2, v_3$ son linealmente independientes, se tiene que $a=0$,
        $a+b = 0$ y $a+b+c = 0$. Encontrando $a,b$ y $c$ se tiene que $a=0=b=c$.
        Lo cual muestra que los vectores $\{ v_1 + v_2 + v_3, v_2 + v_3, v_3 \}$ son 
        linealmente independientes y pueden representar a cualquier vector $y \in V$ como
        $y = d(v_1 + v_2 + v_3) + e(v_2 + v_3) + fv_3$.
    \end{proof}
\end{enumerate}

\item Sean $V$ un espacio vectorial y $S_1, S_2 \subseteq V$.
Demuestre que = $\langle S_1 \cup S_2 \rangle = \langle S_1 \rangle + \langle S_2 \rangle$.
\begin{proof}\hfill
\begin{enumerate}
    \item[$\subseteq)$] Sea $v \in \langle S_1 \cup S_2 \rangle$. Entonces $v$ puede ser escrito
    como una combinación lineal de los vectores en $S_1 \cup S_2$ como
    $v = \sum_i a_i x_i + \sum_j b_j y_y$, donde $a_i, b_j \in F$ y $x_i \in S_1$ y $y_i \in S_2$.
    Como $\sum_i a_i x_i \in \langle S_1 \rangle$ y $\sum_j b_i y_i \in \langle S_2 \rangle$,
    entonces $v \in \langle S_1 \rangle + \langle S_2 \rangle$.

    \item[$\supseteq)$] Sea $v \in \langle S_1 \rangle + \langle S_2 \rangle$, por definición
    se tiene que $v = \sum_i a_i x_i + \sum_j b_j y_y$ donde $a_i, b_j \in F$ y $x_i \in S_1$
    y $y_i \in S_2$. Lo cual es una combinación lineal de los vectores de $S_1 \cup S_2$.
    Por tanto $v \in \langle S_1 \cup S_2 \rangle$.
    
\end{enumerate}
\end{proof}

\item Considere el espacio vectorial $\mathscr{F} (\R, \R)$ con la suma y producto por un escalar
usuales de funciones. Se definen los subconjuntos $W_1, W_2 \subseteq V$ dados por:
\[
    W_1 = \{ f : \R \to \R \ | \ \text{es una función par} \} \quad \text{ y } \quad
    W_2 = \{ f : \R \to \R \ | \ \text{es una función impar} \}
\]
Resuelva los siguientes incisos justificando adecuadamente cada una de sus afirmaciones:
\begin{enumerate}[label=(\alph*)]
    \item Demuestre que $W_1$ y $W_2$ son subespacios vectoriales de $\mathscr{F} (\R, \R)$.
    \begin{proof}\hfill
    \begin{itemize}
        \item $W_1$ es un subespacio de $\mathscr{F} (\R, \R)$.
        
        Sabemos que un función $f: \R \to \R$ es par si $f(x) = -x$ para todo $x \in R$.

        Definamos la función $0: \R \to \R = 0$ como $0(x) = 0$ para todo $x \in \R$. Como $0 \in W_1$
        porque $0(x) = 0 = 0(-x)$ para toda $x \in \R$

        Sabemos que para cualquier $a \in \R$ y $f \in \R$ se tiene que,
        $(a \cdot f)(-x) = a \cdot (f(-x)) = a \cdot (f(x)) = (a \cdot f) (x)$, porque $f$ es par.
        Teniendo así que $a \cdot f$ es par, lo que significa que $a \cdot f \in W_1$.

        De forma similar sabemos que $f, g \in \R$ se tiene que
        $(f+g)(-x) = f(-x) + g(-x) = f(x) + g(x) = (f+g)(x)$, como $f, g$ son pares entonces se tiene
        que $f + g \in W_1$.

        \item $W_2$ es un subespacio de $\mathscr{F} (\R, \R)$.
        
        Sabemos que un función $f: \R \to \R$ es impar si $f(-x) = -f(x)$ para todo $x \in R$.
        
        Definamos la función $0: \R \to \R = 0$ como $0(x) = 0$ para todo $x \in \R$. Como $0 \in W_2$
        porque $0(-x) = 0 = -0(x)$ para toda $x \in \R$.

        Sabemos que para cualquier $a \in \R$ y $f \in \R$ se tiene que,
        $(a \cdot f)(-x) = a \cdot (f(-x)) = a \cdot (-f(x)) = -(a \cdot f) (x)$, porque $f$ es impar.
        Teniendo así que $a \cdot f$ es impar, lo que significa que $a \cdot f \in W_2$.

        De forma similar sabemos que $f, g \in \R$ se tiene que
        $(f+g)(-x) = f(-x) + g(-x) = -f(x) - g(x) = -(f(x) + g(x)) = -(f+g)(x)$, como $f, g$ son
        impares entonces se tiene que $f + g \in W_2$.
    \end{itemize}
    \end{proof}

    \item Demuestre que $\mathscr{F}(\R, \R) = W_1 \oplus W_2$.
    \begin{proof}
        Lo primero que debemos mostrar es que $W_1 \cap W_2 = \{ 0 \}$, tomemos $f \in W_1 \cap W_2$,
        entonces para cada $x \in \R$ se tiene que $f(x) = f(-x)$, pero como $f$ es para también tenemos
        que $f(-x) = -f(x)$ porque $f$ es impar. Lo que juntas significa que $-f(x)=f(x)$ lo cual implica
        que $f(x) = 0$. Pero si $f(x) = 0$ para toda $x \in \R$, entonces $f$ es la función $0$ mostrando
        así que $W_1 \cap W_2 = \{ 0 \}$.

        Para mostrar que $W_1 + W_2 = \mathscr{F} (\R, \R)$ se tiene que para cualquier
        $f \in \mathscr{F} (\R, \R)$ definimos $f_p$ y $f_i$ funciones por
        \[
            f_p(x) = \frac{f(x) + f(-x)}{2} \quad \text{ y } \quad
            f_i(x) = \frac{f(x) - f(-x)}{2}
        \]
        Comprobemos que $f_p \in W_1$ es par,
        \[ f_p(-x) = \frac{f(-x) + f(x)}{2} = \frac{f(x) + f(-x)}{2} = f_p(x)\] 
        Comprobemos que $f_i \in W_2$ es impar,
        \[ f_i(-x) = \frac{f(-x) - f(x)}{2} = \frac{-f(x) + f(-x)}{2} = 
            \frac{f(x) - f(-x)}{2} = -f_i(x)\] 
        También se tiene que,
        \[ (f_p + f_i) (x) =
            \frac{f(-x)+f(x)}{2} + \frac{f(-x) - f(x)}{2} = f(x) \]
        así $f_p + f_i = f$, lo cual muestra que cada función $f$ puede ser escrita como una suma de
        una función par $f_p$ y una impar $f_i$, teniendo así que $W_1 + W_2 = \mathscr{F} (\R, \R)$.

        Por tanto $\mathscr{F}(\R, \R) = W_1 \oplus W_2$.
    \end{proof}
\end{enumerate}

\item Sean $W_1, W_2$ subespacios de un espacio vectorial $V$ tales que $V = W_1 \oplus W_2$. Sean
$\beta_1$ y $\beta_2$ bases de $W_1$ y $W_2$, respectivamente. Resuelva los siguientes incisos
justificando adecuadamente cada una de sus afirmaciones:
\begin{enumerate}[label=(\alph*)]
    \item $\beta_1 \cap \beta_2 = \emptyset$
    \begin{proof}
        Como $V = W_1 \oplus W_2$, tenemos que $W_1 \cap W_2 = \{ 0 \}$. Sean
        $\beta_1 = \{ v_1, \ldots, v_m \}$ y $\beta_2 = \{ w_1, \ldots, w_m \}$. Notemos que los
        vectores de la base $v_1, \ldots, v_m  \in W_1$ y $w_1, \ldots, w_m \in W_2$ son todos no
        ceros, porque son linealmente independientes. Por tanto $\beta_1 \cap \beta_2 = \emptyset$.
    \end{proof}

    \item $\beta_1 \cup \beta_2$ es una base de $V$
    \begin{proof}
        Consideremos $\beta = \beta_1 \cup \beta_2 = \{ v_1, \ldots, v_m, w_1, \ldots, w_m \}$. Los
        vectores de $\beta$ son linealmente independientes, ya que si
        $(a_1v_1 + \cdots + a_nv_m) + (b_1 w_1 + \cdots + b_nw_m) = 0$, entonces el primer vector
        está en $W_1$ y el segundo en $W_2$ así
        $a_1v_1 + \cdots + a_nv_m = b_1 w_1 + \cdots + b_nw_m = 0$. Como $\beta_1, \beta_2$ son bases,
        nos da que $a_1 = \cdots = a_n = 0$ y $b_1 = \cdots = b_n = 0$. Los vectores de $\beta$
        también generan a $V$ porque $V = W_1 \oplus W_2$ lo que significa que cada vector $v \in V$
        puede escribirse como $v = x + y$ con $x \in W_1$ y $t \in W_2$. Pero $\beta_1, \beta_2$ son
        bases para $W_1$ y $W_2$ respectivamente, entonces
        $v = x + y = a_1v_1 + \cdots + a_nv_m = b_1 w_1 + \cdots + b_nw_m$ para algunos coeficientes
        $a_i$ y $b_j$. Por tanto $\beta_1 \cup \beta_2$ son una base pasa $V$.
    \end{proof}
\end{enumerate}

\item Considere la función $T : \mathds{P}_3(\R) \to \mathds{P}_4(\R)$ dada por
$T(f(x)) = 5f'(x) - \int_{0}^{x} f(t)dt$. Resuelva los siguientes incisos justificando adecuadamente
cada una de sus afirmaciones:
\begin{enumerate}[label=(\alph*)]
    \item Demuestre que $T$ es una transformación lineal.
    \begin{proof}
        Sean $f,g \in \mathds{P}_3(\R)$ y $a \in \R$, entonces,
        \begin{align*}
            T((af+g)(x))
                &= 5(af+g)'(x) - \int_{0}^{x} (af+g)(t)dt\\
                &= 5(af'(x)+g(x)) - \int_{0}^{x} (af(t) + g(t)) dt\\
                &= 5(af'(x)+g(x)) - a\int_{0}^{x} f(t) - \int_{0}^{x} g(t) dt\\
                &= a(5f'(x) - \int_{0}^{x} f(t)dt) + 5g'(x) - \int_{0}^{x} g(t) dt\\
                &= aT(f(x)) + T(g(x))
        \end{align*}
        Por lo tanto $T$ es una transformación lineal.
    \end{proof}

    \item Calcule $dim (\mathds{P}_3(\R))$ y $dim (\mathds{P}_4(\R))$
    \begin{proof}
        Tomemos $\beta = \{ 1, x, x^2, x^3 \}$ y $\gamma = \{ 1, x, x^2, x^3, x^4 \}$ las bases
        estándares de $\mathds{P}_3(\R)$ y $\mathds{P}_4(\R)$ respectivamente, pero por definición
        la dimesión de un espacio vectorial es la cardinalidad de sus bases. Por tanto
        $dim (\mathds{P}_3(\R)) = 4$ y $dim (\mathds{P}_4(\R)) = 5$.
    \end{proof}

    \item Encuentre los subespacios $N(T)$ y $R(T)$.
    \begin{proof}
        Tomemos la base estándar $\beta = \{ 1, x, x^2, x^3 \}$ de $dim (\mathds{P}_3(\R))$ donde
        \[ R(T) = \langle \{ T(1), T(x), T(x^2), T(x^3) \} \rangle =
            \langle \{ -x, 5-\frac{x^2}{2}, 10x -\frac{x^3}{3}, 15x^2 - \frac{x^4}{4} \}\rangle \]
        Como $\{ -x, 5-\frac{x^2}{2}, 10x -\frac{x^3}{3}, 15x^2 - \frac{x^4}{4} \}$ son linealmente
        independientes, entonces $dim(R(T)) = 4$.
        
        Usando el teorema de la dimesión $4 = R(T) + N(T) = 4 + N(T) = 4 + 0$, se concluye que
        $N(T) = \{ 0 \}$.
    \end{proof}

    \item Encuentre bases para $N(T)$ y $R(T)$, ¿cuáles son sus dimensiones?
    \begin{proof}
        Para $N(T)$ su base es
        $\beta = \{ T(1), T(x), T(x^2), T(x^3) \} = \{ -x, 5-\frac{x^2}{2}, 10x -\frac{x^3}{3}, 15x^2 - \frac{x^4}{4} \}$
        y $dim(N(T))= 4$. Por el inciso anterior $N(T) = \{ 0 \}$ y $dim(N(T)) = 0$.
    \end{proof}

    \item ¿La transformación lineal $T$ es inyectiva?
    \begin{proof}
        Sí porque si $T$ es lineal si y sólo si $N(T) = \{ 0 \}$, pero por el inciso $c)$
        se tiene que $N(T) = \{ 0 \}$.
    \end{proof}
\end{enumerate}

\end{enumerate}

%\begin{figure}[H]
%    \centering
%    \includegraphics[scale=0.5]{gato.jpg}
%    \caption{Solo me faltó la 6 :(}
%\end{figure}

%%%%%%%%%%%%%%%%%%%%%%%%%%%%%%%%%%%%%%%%%%%%%%%%%%%%%%%%%%%%%%%%%%%%%%%%%%%%%%%%%%%%%%%%%


%%%%%%%%%%%%%%%%%%%%%%%%%%%%%%%%%%%%%%%%%%%%%%%%%%%%%%%%%%%%%%%%%%%%%%%%%%%%%%%%%%%%%%%%%

%%%%%%%%%%%%%%%%%%%%%%%%%%%%%%%%%%%%%%%%%%%%%%%%%%%%%%%%%%%%%%%%%%%%%%%%%%%%%%%%%%%%%%%%%

\end{document}